\documentclass [11pt, a4paper]{article}

\usepackage[left=1.5cm, text={18cm, 25cm}, top = 2.5cm]{geometry}
\usepackage{times}
\usepackage{amsmath}
\usepackage{amsthm}
\usepackage{amsfonts}
\usepackage[czech]{babel}
\usepackage[utf8]{inputenc}
\usepackage[IL2]{fontenc}
\usepackage{verbatim}

\newtheorem{definice}{Definice}
\newtheorem{věta}{Věta}

\begin{document}
\begin{center}
\Huge
\textsc{Fakulta informačních technologií\\
Vysoké učení technické v Brně}
\\[84mm]
\LARGE Typografie a publikování -- 2. projekt\\
Sazba dokumentů a matematických výrazů
\vfill
\end{center}
{\LARGE 2018 \hfill
Sabína Gregušová (xgregu02)}

\thispagestyle{empty}
\clearpage
\setcounter{page}{1}

\begin{twocolumn}
\section*{Úvod}
V této úloze si vyzkoušíme sazbu titulní strany, matematických
vzorců, prostředí a dalších textových struktur obvyklých
pro technicky zaměřené texty (například rovnice \eqref{eq:first}
nebo Definice \ref{def:def_1} na strane \pageref{def:def_1}). Rovněž si vyzkoušíme používání
odkazů \texttt{\textbackslash ref} a \texttt{\textbackslash pageref}.
\par Na titulní straně je využito sázení nadpisu podle optického
strědu s využitím zlatého řezu. Tento postup byl
probírán na přednášce. Dále je použito odřádkování se
zadanou relativní velikostí 0.4em a 0.3em.

\section{Matematický text}
Nejprve se podíváme na sázení matematických symbolů a výrazů v plynulém textu včetně sazby definic a vět s vy\-užitím
balíku \texttt{amsthm}. Rovnež použijeme poznámku pod
čarou s použitím příkazu \texttt{\textbackslash footnote}. Někdy je vhod\-né
použít konstrukci {\verb ${}$ }, která říká, že matematický text nemá být zalomen.
\begin{definice} \label{def:def_1}
\emph{Turingův stroj} (TS) je definován jako šestice
tvaru $M = (Q, \Sigma, \Gamma, \delta , q_0, q_F)$, kde:
\begin{itemize}
\item $Q$ je konečná množina \emph{vnitřních (řídicích) stavů},
\item $\Sigma$ je konečná množina symbolů nazývaná \emph{vstupní abeceda}, $\Delta\notin\Sigma$,
\item $\Gamma$ je konečná množina symbolů, $\Sigma\subset\Gamma$, $\Delta\in\Gamma$, nazývaná \emph{pásková abeceda},
\item $\delta:(Q\backslash\{q_F\})\times\Gamma\rightarrow Q\times(\Gamma\cup\{L,R\}),\text{kde }L,R\notin\Gamma$, je parciální \emph{přechodová funkce},
\item $q_0$ je \emph{počáteční stav}, $q_0\in Q_a$
\item $q_F$ je \emph{koncový stav}, $q_F\in Q$.
\end{itemize}
\end{definice}
Symbol $\Delta$ značí tzv. \emph{blank} (prázdný symbol), který se vyskytuje na místech pásky, která nebyla ješte použita (může ale být na pásku zapsán i později).

\emph{Konfigurace pásky} se skládá z nekonečného řetězce,
který reprezentuje obsah pásky a pozice hlavy na tomto
řetězci. Jedná se o prvek množiny $\{\gamma \Delta^\omega \mid \gamma \in \Gamma^*\}\times\mathbb{N}$.\footnote{Pro libovolnou abecedu $\Sigma$ je $\Sigma^\omega$ množina všech \emph{nekonečných} řetězců nad $\Sigma$, tj. nekonečných posloupností symbolů ze $\Sigma$. Pro připomenutí: $\Sigma^*$ je množina všech \emph{konečných} řetězců nad $\Sigma$.} 
\emph{Konfiguraci pásky} obvykle zapisujeme jako $\Delta xyz \underline{z} x\Delta\ldots$ (podtržení značí pozici hlavy). \emph{Konfigurace stroje} je pak
dána stavem řízení a konfigurací pásky. Formálně se jedná o prvek množiny 
$Q\times\{\gamma\Delta^\omega \mid \gamma\in\Gamma^*\} \times\mathbb{N}$.

\subsection{Podsekce obsahující větu a odkaz}
\begin{definice} \label{def:def_2}
\emph{Řetězec} $w$ \emph{nad abecedou $\Sigma$ je přijat TS} $M$ jestliže $M$ při aktivaci z počáteční konfigurace pásky $\underline{\Delta} w\Delta$\ldots a počátečního stavu $q_0$ zastaví přechodem do koncového stavu $q_F$, tj. $(q_0,\Delta\omega\Delta^\omega,0)\underset{M}{\overset{*}{\vdash}} (q_F,\gamma, n)$ pro nějaké $\gamma\in\Gamma^* \text{ a } n\in\mathbb{N}$.

Množinu $L(M) = \{w\mid w \text{  je přijat TS M}\}\subseteq\Sigma^*$ nazýváme \emph{jazyk přijímaný TS} $M$.
\end{definice}

Nyní si vyzkoušíme sazbu vět a důkazů opět s použitím
balíku \texttt{amsthm}.

\begin{věta}
Třída jazyků, které jsou přijímány TS, odpovídá \emph{rekurzivně vyčíslitelným jazykům.}
\end{věta}

\begin{proof} 
V důkaze vyjdeme z Definice \ref{def:def_1} a \ref{def:def_2}.
\end{proof}

\section{Rovnice a odkazy}
Složitejší matematické formulace sázíme mimo plynulý
text. Lze umístit několik výrazů na jeden řádek, ale pak je třeba tyto vhodně oddělit, například příkazem \texttt{\textbackslash quad}.

$$\sqrt[i]{x^3_i} \quad\text{kde } x_i \text{ je } i \text{-té sudé číslo}\quad  y_i^{2\cdot y_i} \neq y_i^{y_i^{y_i}}$$

V rovnici \eqref{eq:first} jsou využity tři typy závorek s různou explicitně definovanou velikostí.

\begin{align}
\label{eq:first}
x \quad&= \quad\bigg\{\Big(\big[a + b\big]*c\Big)^d\oplus1\bigg\}\\
y \quad&= \quad\lim_{x\to\infty}\frac{\sin^2x + \cos^2x}{\frac{1}{\log_{10}x}} \nonumber
\end{align}

V této věte vidíme, jak vypadá implicitní vysázení limity $\lim_{n\to\infty} f(n)$ v normálním odstavci textu. Podobně je to i s dalšími symboly jako $\sum_{i=1}^n 2^i$ či $\bigcup_{A\in\mathcal{B}}A$. V případe vzorců $\lim\limits_{n\to\infty} f(n)$ a $\sum\limits_{i=1}^n 2^i$ jsme si vynutili méně úspornou sazbu příkazem \texttt{\textbackslash limits}.

\begin{eqnarray}
\int\limits_a^b f(x)\,\mathrm{d}x &=& - \int_b^a g(x)\,\mathrm{d}x\\
\overline{\overline{A\vee B}} &\Leftrightarrow& \overline{\overline{A}\wedge \overline{B}}
\end{eqnarray}
\section{Matice}
Pro sázení matic se velmi často používá prostředí \texttt{array} a závorky (\texttt{\textbackslash left, \textbackslash right}).

$$\left( \begin{array}{ccc}
a + b & {\widehat {\xi + \omega}} & {\hat \pi}\\
{\vec a} & \overleftrightarrow{AC} & \beta
\end{array} \right)
= 1 \iff \mathbb{Q} = \mathbb{R}
$$

$$ 
\text{A} = \left\| \begin{array}{cccc}
a_{11} & a_{12} & \cdots & a_{1n} \\
a_{21} & a_{22} & \cdots & a_{2n} \\
\vdots & \vdots & \ddots & \vdots \\
a_{m1} & a_{m2} & \cdots & a_{mn} \end{array} \right\|
= 
\left| \begin{array}{cc}
t & u \\
v & w \\
\end{array} \right|
= tw - uv
$$


Prostředí \texttt{array} lze úspešne využít i jinde.

$$
\begin{pmatrix}
n \\
k
\end{pmatrix}=\left\{
\begin{array}{ll}
\frac{n!}{k!(n-k)!} & \text{pro } 0\leq k\leq n \\
0 & \text{pro } k < 0 \text{ nebo } k > n
\end{array} \right. $$

\section{Závěrem}
V případě, že budete potřebovat vyjádřit matematickou
konstrukci nebo symbol a nebude se Vám dařit jej nalézt
v samotném \LaTeX u, doporučuji prostudovat možnosti balíku maker \AmS-\LaTeX.
\end{twocolumn}

\end{document}